\documentclass{article}

\usepackage{amsmath}
\usepackage[utf8]{inputenc}
\usepackage[russian]{babel}
\usepackage[a4paper, margin=1in]{geometry}
\usepackage{xcolor}
\usepackage{colortbl}
\usepackage{graphicx}
\usepackage{tikz}
\usepackage{enumitem}

\setlength{\parindent}{0pt}
\setlength{\parskip}{1pt}

\title{Домашнее задание №1}
\author{Горский Кирилл, группа 3384. Вариант 5}
\date{\today}

\begin{document}

\maketitle
\pagebreak

\subsubsection*{Исходные данные (первый квадрант)}
\paragraph{}
Вершины \(a\) и \(b\):

\input{input_vertices.tex}

\paragraph{}
Грани:

\input{input_faces.tex}

\paragraph{}
Если визуализировать то выглядит так:

\begin{center}
    \includegraphics[width=0.5\textwidth]{input_poly.png}
\end{center}

\pagebreak

\subsubsection*{Построение многогранника}

\paragraph{}
Отражаем вершины во все остальные квадранты, получаем 48 вершин:

\input{all_vertices.tex}

\paragraph{}
Уникальных среди них 18:

\input{unique_vertices.tex}

\paragraph{}
Добавляем симметричные грани, получается суммарно 32 грани:

\input{all_faces.tex}

\paragraph{}
Если визуализировать то выглядит так:

\begin{center}
    \includegraphics[width=0.4\textwidth]{poly.png}
\end{center}

\pagebreak
\paragraph{}
Для проверки на выпуклость нужны уравнения плоскостей.
Для этого для каждой грани найдём нормаль плоскости $n = (n_x, n_y, n_z)$ с помощью векторного произведения двух её рёбер,
а также выберем $p = (x_0, y_0, z_0)$ произвольным образом из списка вершин, которыми эта грань задается. 
В случае, если нормаль смотрит в сторону нуля ($\langle p, n \rangle < 0$),
то перевернём её. Уравнение плоскости в таком случае будет:

\[n_x(x-x_0)+n_y(y-y_0)+n_z(z-z_0) = 0 \]

\[n_x x - n_x x_0 + n_y y - n_y y_0 + n_z z - n_z z_0 = 0 \]

\[n_x x + n_y y + n_z z - n_x x_0 - n_y y_0 - n_z z_0 = 0 \]

\[
\begin{cases}
    Ax + By + Cz + D = 0 \\
    A = n_x, B = n_y, C = b_z \\
    D = - n_x x_0 - n_y y_0 - n_z z_0
\end{cases}
\]

Получаем такие уравнения плоскостей:

\input{planes.tex}

\begin{center}
    \includegraphics[width=0.3\textwidth]{poly_normals_1.png}
    \includegraphics[width=0.3\textwidth]{poly_normals_2.png}
    \includegraphics[width=0.3\textwidth]{poly_normals_3.png}
\end{center}

\end{document}
