\documentclass{article}

\usepackage{amsmath}
\usepackage[utf8]{inputenc}
\usepackage[russian]{babel}
\usepackage[a4paper, margin=1in]{geometry}
\usepackage{xcolor}
\usepackage{colortbl}
\usepackage{graphicx}
\usepackage{tikz}
\usepackage{enumitem}

\setlength{\parindent}{0pt}
\setlength{\parskip}{1pt}

\title{Домашнее задание №1}
\author{Горский Кирилл, группа 3384. Вариант 5}
\date{\today}

\begin{document}

\maketitle
\pagebreak

\subsubsection*{Исходные данные (первый квадрант)}
\paragraph{}
Вершины \(a\) и \(b\):

\input{input_vertices.gen.tex}

\paragraph{}
Грани:

\input{input_faces.gen.tex}

\paragraph{}
Можно нарисовать эти грани:

\begin{center}
    \includegraphics[width=0.5\textwidth]{input_poly}
\end{center}

\pagebreak

\subsubsection*{Построение многогранника}

\paragraph{}
Отражаем вершины во все остальные квадранты, получаем 48 вершин:

\input{all_vertices.gen.tex}

\paragraph{}
Уникальных среди них 18:

\input{unique_vertices.gen.tex}

\paragraph{}
Добавляем симметричные грани, получается суммарно 32 грани:

\input{all_faces.gen.tex}

\paragraph{}
В качестве самопроверки полученный многогранник можно нарисовать:

\begin{center}
    \includegraphics[width=0.4\textwidth]{poly}
\end{center}

\pagebreak
\subsubsection*{Проверка на выпуклость}
Для проверки на выпуклость нужны уравнения плоскостей.
Для этого для каждой грани найдём нормаль плоскости $n = (n_x, n_y, n_z)$
с помощью векторного произведения двух её рёбер,
а также выберем $p = (x_0, y_0, z_0)$ произвольным образом из списка вершин, которыми эта грань задается. 
В случае, если вычисленная нормаль смотрит в сторону нуля ($\langle p, n \rangle < 0$),
то перевернём её. Уравнение плоскости в таком случае будет:

\[n_x(x-x_0)+n_y(y-y_0)+n_z(z-z_0) = 0 \]
\[n_x x - n_x x_0 + n_y y - n_y y_0 + n_z z - n_z z_0 = 0 \]
\[n_x x + n_y y + n_z z - n_x x_0 - n_y y_0 - n_z z_0 = 0 \]

\[
\begin{cases}
    Ax + By + Cz + D = 0 \\
    A = n_x, B = n_y, C = n_z \\
    D = - n_x x_0 - n_y y_0 - n_z z_0
\end{cases}
\]

Получаем такие уравнения плоскостей:

\input{planes.gen.tex}

\paragraph{}
Можно также нарисовать рисунок, чтобы лишний раз убедиться, что нормали смотрят наружу:

\begin{center}
    \includegraphics[width=0.5\textwidth]{poly_normals}
\end{center}

\pagebreak
Теперь координаты каждой вершины подставим в уравнения плоскостей:

\input{plane_checks.gen.tex}

\paragraph{}
Полученные значения не больше нуля, значит все точки лежат в отрицательных полуплоскостях,
значит многогранник выпуклый.

\pagebreak
\subsubsection*{Норма Минковского}

\paragraph{}
Норма Минковского: $\left \lVert x \right \rVert = \inf \{ \lambda > 0: \frac{x}{\lambda} \in W \}$.
Другими словами, нам нужно найти такую точку между началом координат и $x$, чтобы она лежала на границе
множества. Соотношение между $x$ и этой точкой будет нормой Минковского.

\paragraph{}
Барицентрические координаты -- это координаты точки в биортогональном базисе грани.

\paragraph{}
Вычислим барицентрические координаты для каждой грани первого квадранта.
\input{change_var_a.gen}:

\input{bary_table_a.gen.tex}

\paragraph{}
Точка $a$ лежит внутри только одного из конусов. Чтобы получить точку, лежащую на треугольнике,
нужно с помощью деления привести сумму $k_1 + k_2 + k_3$ к единице.
Следовательно, $\lambda = k_1 + k_2 + k_3$.

\begin{center}
    \includegraphics[width=0.5\textwidth]{minkowski_norm_a}
\end{center}

\paragraph{}
Таким образом, \input{minkowski_norm_a.gen.tex}

\paragraph{}
Почему вычисление нормы можно выполнять в первом квадранте? Перевод точки из первого квадранта в другой заключается в
покомпонентном умножении на $(\pm 1, \pm 1, \pm 1)$.

\pagebreak
\paragraph{}

Аналогичные вычисления можно проделать и для точки $b$.
\input{change_var_b.gen}:

\input{bary_table_b.gen.tex}

\begin{center}
    \includegraphics[width=0.5\textwidth]{minkowski_norm_b}
\end{center}

\paragraph{}
Таким образом, \input{minkowski_norm_b.gen.tex}.


\pagebreak
\paragraph{}
Аналогичные вычисления можно проделать и для точки $a + b$.
\input{change_var_sum.gen}:

\input{bary_table_sum.gen.tex}

\begin{center}
    \includegraphics[width=0.5\textwidth]{minkowski_norm_sum}
\end{center}

\paragraph{}
Таким образом, \input{minkowski_norm_sum.gen.tex}.

\pagebreak
\paragraph{}
Получили, что 
\input{minkowski_norm_a.gen.tex},
\input{minkowski_norm_b.gen.tex},
\input{minkowski_norm_sum.gen.tex}.

\paragraph{}
\input{triangle_check.gen.tex}.

\begin{center}
    \includegraphics[width=0.5\textwidth]{triangle}
\end{center}

\end{document}
