\documentclass{article}

\usepackage{amsmath}
\usepackage[utf8]{inputenc}
\usepackage[russian]{babel}
\usepackage[a4paper, margin=1in]{geometry}
\usepackage{xcolor}
\usepackage{colortbl}
\usepackage{graphicx}
\usepackage{tikz}
\usepackage{enumitem}

\setlength{\parindent}{0pt}
\setlength{\parskip}{1pt}

\title{Домашнее задание №1}
\author{Горский Кирилл, группа 3384. Вариант 5}
\date{\today}

\begin{document}

\maketitle
\pagebreak

\subsubsection*{Исходные данные (первый квадрант)}
\paragraph{}
Вершины \(a\) и \(b\):

\input{input_vertices.gen.tex}

\paragraph{}
Грани:

\input{input_faces.gen.tex}

\paragraph{}
Можно нарисовать эти грани:

\begin{center}
    \includegraphics[width=0.5\textwidth]{input_poly}
\end{center}

\pagebreak

\subsubsection*{Построение многогранника}

\paragraph{}
Отражаем вершины во все остальные квадранты, получаем 48 вершин:

\input{all_vertices.gen.tex}

\paragraph{}
Уникальных среди них 18:

\input{unique_vertices.gen.tex}

\paragraph{}
Добавляем симметричные грани, получается суммарно 32 грани:

\input{all_faces.gen.tex}

\paragraph{}
В качестве самопроверки полученный многогранник можно нарисовать:

\begin{center}
    \includegraphics[width=0.4\textwidth]{poly}
\end{center}

\pagebreak
\subsubsection*{Проверка на выпуклость}
Для проверки на выпуклость нужны уравнения плоскостей.
Для этого для каждой грани найдём нормаль плоскости $n = (n_x, n_y, n_z)$ с помощью векторного произведения двух её рёбер,
а также выберем $p = (x_0, y_0, z_0)$ произвольным образом из списка вершин, которыми эта грань задается. 
В случае, если вычисленная нормаль смотрит в сторону нуля ($\langle p, n \rangle < 0$),
то перевернём её. Уравнение плоскости в таком случае будет:

\[n_x(x-x_0)+n_y(y-y_0)+n_z(z-z_0) = 0 \]
\[n_x x - n_x x_0 + n_y y - n_y y_0 + n_z z - n_z z_0 = 0 \]
\[n_x x + n_y y + n_z z - n_x x_0 - n_y y_0 - n_z z_0 = 0 \]

\[
\begin{cases}
    Ax + By + Cz + D = 0 \\
    A = n_x, B = n_y, C = n_z \\
    D = - n_x x_0 - n_y y_0 - n_z z_0
\end{cases}
\]

Получаем такие уравнения плоскостей:

\input{planes.gen.tex}

\paragraph{}
Можно также нарисовать рисунок, чтобы лишний раз убедиться, что нормали смотрят наружу:

\begin{center}
    \includegraphics[width=0.5\textwidth]{poly_normals}
\end{center}

\pagebreak
Теперь координаты каждой вершины (из списка уникальных вершин, которых 18 штук) подставим
в уравнения всех плоскостей:

\input{plane_checks.gen.tex}

\paragraph{}
Полученные значения не больше нуля, значит все точки лежат в отрицательных полуплоскостях,
значит многогранник выпуклый.

\pagebreak
\subsubsection*{Норма Минковского}

\paragraph{}
Для каждой грани найдём расстояние от центра координат до плоскости грани и
расстояние от центра координат до точки $a$ по направлению нормали плоскости.
Чтобы получить точку, лежащую в плоскости грани, поделим её на расстояние от центра
до неё самой и умножим на расстояние от центра до плоскости грани:

\[\hat P = P \cdot \frac{\langle P_0, n \rangle}{\langle a, n \rangle}, \]
где $P_0$ -- некоторая точка, лежащая в плоскости грани.

\paragraph{}
Чтобы найти координаты точки $P$ в барицентрических координатах треугольника $ABC$ нужно:
\begin{itemize}
    \item Найти площадь треугольника $ABC$ ($S_{ABC}$)
    \item Найти площади треугольников, которые образует точка $P$ с вершинами треугольника $ABC$
          ($S_{PBC}$, $S_{PCA}$, $S_{PBA}$)
    \item $k_1 = S_{PBC} / S_{ABC}$
    \item $k_2 = S_{PCA} / S_{ABC}$
    \item $k_3 = S_{PBA} / S_{ABC}$
\end{itemize}

\paragraph{}
Если точка лежит внутри треугольника, то внутренние треугольники не будут пересекаться и сумма $k_1+k_2+k_3$ будет равна 1.

\paragraph{}
Площадь треугольника вычисляется как проекция векторного произведения двух рёбер треугольника на его нормаль.
Например, для треугольника $ABC$ площадь будет:

\[ S_{ABC} = \langle n, (B-A) \times (C-A) \rangle \]

\paragraph{}
На всякий случай скалярное произведение будем брать по модулю, мало ли нормаль была перевернута.

\paragraph{}
Вычислим барицентрические координаты для каждой грани

\input{bary_table_a.gen.tex}

\pagebreak
\paragraph{}
Среди всех $\hat P$ внутри треугольника лежат только две.
Прямая, проходящая через точку $a$ и центр координат, проходит многоугольник насквозь.
По определению нас интересует точка с наибольшей $\lambda$. Можно её нарисовать:

\begin{center}
    \includegraphics[width=0.5\textwidth]{minkowski_norm_a}
\end{center}

\paragraph{}
Таким образом, \input{minkowski_norm_a.gen.tex}

\pagebreak
\paragraph{}
Аналогичные вычисления можно проделать и для точки $b$:

\input{bary_table_b.gen.tex}

\begin{center}
    \includegraphics[width=0.5\textwidth]{minkowski_norm_b}
\end{center}

\paragraph{}
Красным цветом выделены строчки, соответствующие граням, нормаль которых оказалась перпендикулярна
вектору из начала координат в точку $b$. На плоскости этих граней точку $b$ спроецировать впринципе нельзя. 

\paragraph{}
Таким образом, \input{minkowski_norm_b.gen.tex}.

\end{document}
